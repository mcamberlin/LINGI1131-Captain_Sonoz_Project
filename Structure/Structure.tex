%----------------------------------------------------------------------------------------
%	MERLIN CAMBERLIN ! ALL RIGHTS RESERVED
%----------------------------------------------------------------------------------------

%----------------------------------------------------------------------------------------
%	PACKAGES ET AUTRES CONFIGURATIONS
%----------------------------------------------------------------------------------------
\usepackage[french]{babel} 
\usepackage[utf8]{inputenc}
%\usepackage[latin1]{inputenc}
\usepackage[T1]{fontenc}

\usepackage{supertabular}% pour séparer un tableau lorsqu'il est sur plusieurs pages
\usepackage{ulem} %pour souligner
\usepackage[bottom]{footmisc} % pour mettre les notes de bas de page bien en bas
%%% Pour les couleurs
\usepackage{pdflscape}
\usepackage[svgnames]{xcolor}
\usepackage{colortbl}%
    \newcommand{\myrowcolour}{\rowcolor[gray]{0.925}}
\usepackage{booktabs}
\definecolor{lightgray}{gray}{0.5}

%Pour supprimer l'indentation automatique
\usepackage{parskip}
\usepackage{comment} % Pour ajouter des commentaires sur plusieurs lignes avec \begin{comment} ... \end{comment}



%%% Pour la mise en page par défaut
\usepackage{fullpage}
\usepackage[left=2cm,right=2cm,top=2cm,bottom=2cm]{geometry}
\usepackage{lastpage}%Donne la dernière page

\usepackage{setspace} % gérer l'interligne

%%% Pour écrire du texte sur plusieurs colonnes
\usepackage{multicol}
\usepackage{enumitem} % why not, item
\frenchbsetup{StandardLists=true} % à inclure si on utilise \usepackage[french]{babel} pour des bubulles

%%% Pour les circuits électriques
\usepackage{tikz}
\usepackage{circuitikz}


%%% Pour créer un en-tête et un pied de page
\usepackage{fancyhdr} % Needed to define custom headers/footers
\pagestyle{fancy} % Enables the custom headers/footers

\renewcommand{\headrulewidth}{0.pt} % No header rule
\renewcommand{\footrulewidth}{0.pt} % Thin footer rule

\renewcommand{\sectionmark}[1]{\markboth{#1}{}} % Removes the section number from the header when \leftmark is used

%\nouppercase\leftmark % Add this to one of the lines below if you want a section title in the header/footer

% Headers
\lhead{} % Left header
\chead{} % Center header - currently printing the article title
\rhead{} % Right header

% Footers
%\lfoot{\includegraphics[scale=0.18]{Images/exemple.jpg}}
%\cfoot{\color{lightgray}{\nouppercase\leftmark }% Add this to one of the lines below if you want a section title in the header/footer} % Center footer
\fancyfoot[CE,CO]{\color{lightgray} \nouppercase \leftmark}
\fancyfoot[LE,RO]{\thepage}
%\rfoot{\footnotesize Page \thepage\ sur \pageref{LastPage}} % Right footer, "Page 1 of 2"
\usepackage{etoolbox}
\patchcmd{\chapter}{\thispagestyle{plain}}{\thispagestyle{fancy}}{}{}


%%% Bibliographie
\usepackage[sorting=none]{biblatex} 
    \addbibresource{bibliographie.bib}
\usepackage[autostyle=true]{csquotes} 

%%% Pour insérer des mathématiques
\usepackage{amsmath,amsfonts,amsthm,amssymb}
\usepackage{eqnarray}
\usepackage{systeme} % pour faire des systèmes d'équations
\usepackage{numprint} % Pour que les chiffres s'écrivent seul en notation scientifique


%%% Pour insérer des caractères spéciaux
\usepackage{gensymb}    % Pour insérer le signe degré
\usepackage{eurosym}    % Pour insérer le signe €

%%% Pour insérer des images
\usepackage{graphicx}
\usepackage{float}          % Pour obliger une position d'une figure \begin{figure}[H]
\usepackage[tt]{titlepic}   % Package pour insérer une image dans la première page
\usepackage{subfigure}  % Permet d'avoir des sous-images
\usepackage{wrapfig} % Permet d'avoir du texte autour d'une image

%%% Pour insérer un pdf
\usepackage{pdfpages}
%%% Pour insérer des URL
\usepackage{url}

%%% Package pour insérer du code 
\usepackage{listings}
\usepackage{color}
 
\definecolor{codegreen}{rgb}{0,0.6,0}
\definecolor{codegray}{rgb}{0.5,0.5,0.5}
\definecolor{codepurple}{rgb}{0.58,0,0.82}
\definecolor{backcolour}{rgb}{0.95,0.95,0.92}
 
\lstdefinestyle{styleMerlin}{
    language = C,% ou java
    backgroundcolor=\color{backcolour},  
    commentstyle=\color{codegreen},%\color[rgb]{0.133,0.545,0.133},
    %identifierstyle=\ttfamily,
    keywordstyle=\color{magenta},%\color[rgb]{0,0,1},
    numberstyle=\tiny\color{codegray},
    stringstyle=\color{codepurple},%\color[rgb]{0.627,0.126,0.941},
    basicstyle=\footnotesize,%\scriptsize
    breakatwhitespace=false,         
    breaklines=true,                 
    captionpos=b,                    
    keepspaces=true,                 
    numbers=left,                    
    numbersep=5pt,                  
    showspaces=false,                
    showstringspaces=false,
    showtabs=false,                  
    tabsize=2 % ajouter la virgule ici et l'enlever à la derniere ligne
    %upquote=true,
    %aboveskip={1.5\baselineskip},
    %columns=fullflexible,
    %extendedchars=true,
}
\lstset{style=styleMerlin}
%\begin{lstlisting}[language=C, caption=Python example]
%#include <stdio.h>
% ...
%\end{lstlisting}
%----------------------------------------------------------------------------------------
%	MERLIN CAMBERLIN ! ALL RIGHTS RESERVED
%----------------------------------------------------------------------------------------